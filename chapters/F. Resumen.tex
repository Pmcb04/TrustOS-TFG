\chapter*{Resumen}
\thispagestyle{chapterpage}

La ultima vez que fui al supermercado, me compré café 100\% colombiano, o al menos eso me aseguraba la pegatina del envase... Desde entonces estuve pensando en todo el proceso que se había desarrollado con el café desde su supuesto origen hasta que llegó a mis manos. La recogida del café, su secado, el posterior envasado, hasta la venta al supermercado. Pasando por supuesto por el transporte desde el otro lado del mundo y las medidas de calidad del producto.

A todo este proceso se llama trazabilidad de un producto, el cual podemos definir como una serie de pasos para transformar materias primas o productos prefabricados en productos terminados.

Si miras a tu alrededor podrás ver una infinidad de productos que han tenido una trazabilidad hasta llegar a tus manos (y por que no seguirán teniendo más cuando termines de usarlos). Sin ir más lejos el movil o portatil que estas utilizando para leer esto tiene una trazabilidad.

Pero volviendo al café que me compré el otro día, ¿qué pasa con los usuarios? ¿Cómo sabemos que país de origen ha tenido el producto? ¿Materiales fabricados? ¿Es nuevo o usado?

Para responder a estas preguntas se ha realizado este proyecto, para dar una solución a los usuarios y que conozcan las respuestas a todas estas preguntas utilizando tecnología puntera como es Blockchain, el cual hablaremos más detenidamente en el desarrollo de este trabajo junto con la construcción de un lenguaje de dominio para la generación de una página web que utiliza la API desarrollada por Telefonica y que le han nombrado TrustOS, para poder utilizar la tecnología Blockchain sin saber nada al respecto.

\thispagestyle{empty}

\chapter*{Abstract}
\thispagestyle{empty}
EL RESUMEN EN INGLÉS
